\documentclass{article}

\begin{document}

Title: SHREC: SHarp REConstruction of Isosurfaces

Authors and affiliations:

Objectives: This paper presents an algorithm, SHREC, for reconstruction 
of isosurfaces with sharp 0-dimensional and 1-dimensional features.
It also gives a methodology for analyzing the quality of the reconstruction.

Approach: Algorithm SHREC constructs an isosurface using dual contouring.
It selects a well-separated set of isosurface vertices on the sharp features
and merges vertices in their neighborhoods to generate good representations
of the 0 and 1-dimensional features by mesh vertices and mesh edges.
We measure the quality of the sharp feature reconstruction
by comparing surface normals in the reconstruction to nearby surface normals 
in a corresponding ``ideal'' mesh.

Major novel contributions:
\begin{itemize}
\item An algorithm for constructing isosurfaces with sharp 0-dimensional 
and 1-dimensional features which is measurably better than previous algorithms.
\item A methodology for measuring the quality of reconstruction of
surfaces with sharp 0-dimensional and 1-dimensional features
based on surface normals.
\end{itemize}

Other journal or conference papers:

MergeSharp - MergeSharp is a previous algorithm which also reconstructed
isosurfaces using dual contouring and cube merging.  The algorithm SHREC
significantly improves on MergeSharp, by better generation
of isosurface vertices, better selection of vertices on sharp features,
and better choices in merging.  SHREC can also be applied to scanned data
(e.g., industrial CT data).
MergeSharp has difficulties reconstructing sharp features
without near perfect gradient information.


Closest prior articles:

Varadhan, Krishnan, Kim, Manocha

or

Zhang, Hong and Kaufman.

Four experts:

Kobbelt, Ju, Schaefer, Warren, Scheidegger, Silva.

Associate editors:

Kobbelt, Tao Ju

\end{document}
