
\firstsection{Introduction}

\maketitle

Given a regular grid sampling of a scalar field, $f: \Rthree \rightarrow \R$,
and a scalar value $\sigma$,
we are interested in constructing a good mesh representation
of the level set $f^{-1}(\sigma) = \{x : f(x) = \sigma \}$.
Such a representation is called an {\em isosurface} 
and the scalar value $\sigma$ is called an {\em isovalue}.

When $f$ is smooth,
i.e., when $f$ is continuous and its derivatives are continuous,
a number of algorithms do an excellent job of isosurface construction.
However, if the gradient field of $f$ is discontinuous,
then the problem of isosurface construction is more challenging.
If the level set $f^{-1}(\sigma)$ intersects a gradient discontinuity,
then the level set $f^{-1}(\sigma)$ will have a ``sharp corner''
(a 0-dimensional feature) or a ``sharp edge'' (a 1-dimensional feature)
at that discontinuity.
We are interested in constructing isosurfaces
which do a good job of representing those sharp features.

Isosurfaces are represented by piecewise linear or piecewise smooth meshes,
usually composed of triangles or quadrilaterals.
This underlying mesh should model the sharp features
of the isosurface $\Sigma$.
A 1-dimensionalf feature of $\Sigma$ should be represented by a single, 
connected sequence of mesh edges with similar dihedral angle.
A 0-dimensional feature of $\Sigma$ should be represented by a single
isosurface vertex with similar solid angle.
On the other hand, mesh edges and vertices
representing smooth, low curvature portions of $\Sigma$ should have
dihedral angles near 180 degrees and solid angles near $2 \pi$.

...

Our paper contains the following contributions:
\begin{enumerate}
\item We describe an isosurface generation algorithm
called SHREC which does measurable better than previous algorithms
in constructing isosurfaces with sharp features.
In particular, SHREC does better at generating and selecting vertices 
on sharp features and in avoiding constructing degenerate 
or small angle triangles and flipping triangles.
\item We define the ``Angular Distance'' between two surfaces
and use this distance to evaluate the quality 
of surface reconstruction algorithms.
\end{enumerate}

