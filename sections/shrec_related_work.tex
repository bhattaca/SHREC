

\section{Related Work}
\label{section:related}

The well-known Marching Cubes algorithm~\cite{lc-mchr3-87}
by Lorensen and Cline
constructs isosurface patches within active grid cubes.
The isosurface patches align along their common boundaries.
Because Marching Cubes positions vertices only on grid edges,
never inside grid cubes,
it does a poor job at representing sharp features on isosurfaces.

Dual contouring algorithms construct an isosurface using quadrilaterals
which are dual to bipolar grid edges.
The isosurface vertices are located within the grid cubes,
not on grid edges.
Gibson~\cite{gh-ssqem-97,g-cesng-98} gave the first dual contouring algorithm
which she called surface nets.
Because Gibson's algorithm placed only one isosurface vertex 
in each active cube,
it produced many isosurface edges contained in four quadrilaterals.
Thus, the resulting isosurface was not a manifold.
Nielson~\cite{n-dmc-04} modified Gibson's algorithm
to allow multiple isosurface vertices in active cubes.
The number of vertices in a grid cube $\cb$
corresponds to the number of isosurface patches in $\cb$
produced by the Marching Cubes algorithm.

Nielson's algorithm eliminates most, but not all, 
of the non-manifold problems in dual contouring.
A dual contouring algorithm
for constructing an isosurface which is a always a manifold
is contained in~\cite{Wenger:2013:Isosurfaces}.
The algorithm is essentially Nielson's algorithm
but the number and connectivity of isosurface vertices in grid cubes
is sometimes modified.

In~\cite{l-oslpm-00}, Lindstrom gave an algorithm for locating a point 
on a surface from a set of $n$ tangent planes.
The $n$ tangent give a set of $n$ equations in three unknowns,
described by $M x = b$ where $M$ is an $n \times 3$ matrix 
and $b$ is a column vector of length $n$.
Lindstrom uses the singular valued decomposition (SVD) of $M$
to determine a point close to all the tangent planes.
The SVD of $M$ also indicates whether the point is
on a 0-dimensional (corner) or 1-dimensional (edge) surface feature
or on a smooth portion of the surface
Lindstrom's algorithm is described in more detail 
in Section~\ref{section:loc}.
All the papers for isosurface construction with sharp features
use Lindstrom's algorithm or some variation
to locate points on sharp features.
Most of them also use Lindstrom's algorithm to identify sharp features
and classify them as 0 or 1 dimensional.

Kobbelt et al.~\cite{kbsh-fssev-01} modified Marching Cubes 
to position vertices inside grid cubes when those grid cubes
intersected sharp features.
In addition to scalar values at regular grid vertices,
the algorithm by Kobbelt et al. requires a directed distance field
representing the distance along each axis to the modeled surface.
The algorithm uses the directed distance field
to locate points on sharp features.

Ju et al.~\cite{jlsw-dchd-02,sw-dcss-02} present a dual contouring algorithm
for constructing isosurfaces with sharp features.
Vertices of dual contouring isosurfaces lie inside grid cubes
while isosurface quadrilaterals are dual to grid edges.
Dual contouring algorithms were first described
by Gibson in~\cite{gh-ssqem-97,g-cesng-98}.
In addition to scalar values at regular grid vertices,
the algorithm by Ju et al. requires surface normals
at the intersection of the grid edges and the surface.
The surface normals are used to construct tangent planes
intersecting a grid cube.
Ju et al. apply Lindstrom's algorithm to the tangent planes
to locate an isosurface vertex in each grid cube
and determine if the vertex lies on a sharp 0 or 1 dimensional feature.
Variations on~\cite{jlsw-dchd-02} are given 
in~\cite{zhk-dctps-04,Varadhan:2003:fss}.

Ashida and Bandler~\cite{ab-fpmmo-03}, Ho et al.~\cite{hwco-cmsaf-05}
and Gre{\ss} and Klein~\cite{gk-eretm-04} give algorithms
for constructing multiresolution isosurface with sharp features
using oct-trees or kd-trees.
The isosurface mesh is constructed
by first constructing polygonal curves representing the intersection
of the isosurface and oct-tree or kd-tree cells,
and then connecting an isosurface vertex with those polgyonal curves.
As in~\cite{jlsw-dchd-02},
isosurface locations are computed from input surface normals
using Lindstrom's algorithm.

The Dual Marching Cubes\footnote{Nielson also calls his
dual contouring algorithm ``Dual Marching Cubes''~\cite{n-dmc-04},
but it is totally different from Schaefer and Warren's algorithm.}
algorithm by Schaefer and Warren~\cite{sw-dmcpc-04}
constructs a dual mesh which aligns with sharp features
and then extracts the isosurface from that mesh 
using Marching Cubes.
Vertices of the dual mesh are positioned on sharp features
using Lindstrom's algorithm.
Because the isosurface is extracted using Marching Cubes,
the isosurface will have many "sliver" triangles with small angles.
Dual Marching Cubes reduces the number of "sliver" triangles
by positioning the vertices of the dual grid to lie on the isosurface,
whenever possible.

The algorithms described above 
produce ``sliver'' triangles with very small angles 
along the 1-dimensional features.
A small perturbation of a vertex of such triangles
has a large effect on the triangle normal direction,
so the normal direction of such triangles is almost arbitrary.
The triangle angles may be so small that the triangles are effectively
degenerate with zero area.

With the exception of~\cite{sw-dmcpc-04},
all the algorithms listed above generate isosurface vertices
within active grid cubes.
However, a sharp edge (1-dimensional feature) may intersect 
an inactive grid cube  and a sharp corner (0-dimensional feature) 
may lie in an inactive grid cube.
In such cases, the algorithms produce isosurfaces which contain notches
or truncated corners.
The algorithms could be modified to allow placement of isosurface vertices
in inactive grid cubes,
but at the cost of exacerbating the problem 
of sliver and degenerate triangles.
Placing vertices in inactive grid cubes also introduces a new problem 
of isosurface vertices in the wrong order along 1-dimensional features
creating folds in the mesh.

The MergeSharp algorithm by Bhattaca and Wenger~\cite{bw-cisec-13,bw-erm-13}
attacks the problem of sliver triangles, notches, and folds
by merging grid cubes around features
so that isosurface vertices on features are well-separated 
from each other and from non-feature vertices.
Isosurface vertices are permitted to be placed in inactive grid cubes.

Bhattaca and Wenger analyzed their algorithm by extracting all 
``sharp'' mesh edges with dihedral angle below a threshold ($140^\circ$),
forming a graph (1-skeleton) from those edges
and counting the number of vertices with degrees other than two.
For instance, the 1-skeleton from the sharp mesh edges in the reconstruction
of a cube should have eight vertices with degree three
and no vertices with degrees other than two or three.
The 1-skeleton from the sharp mesh edges in the reconstruction
of a thickened annulus should have no vertices with degree other than two.
By counting the difference between the expected and the actual degree counts,
Bhattaca and Wenger gave a quantitative measure of the performance
of their algorithm.

We mention only a few papers from the large literature
on surface and feature reconstruction from point sets.
Point set data is inherently noisy, so much of the literature
focuses on finding the true position of points on surfaces.
Daniels et al.~\cite{Daniels:2007:Robust},
Fleishman et al.~\cite{fcs-rmlsf-2005}
and Oztireli et al.~\cite{Oztireli2009}
construct local surface patches fitted to local sets of points
and project points onto these surface patches.
Wang et al.~\cite{Wang:2013:Feature}
construct approximations of the tangent planes at the sample points
and project points onto these tangent planes.
Avron et al.~\cite{avron2010L} estimate surface normals
at sample points using convex optimization,
and then reposition the points, again using convex optimization.

The papers cited above focus on correct positioning of surface points
in the presence of sharp features.
The actual construction of the surface mesh is
left to preexisting algorithms.
For constructing the surface mesh,
Oztireli et al. and Wang et al. use Marching Cubes,
Daniels et al. use the advancing front algorithm 
from~\cite{Schreiner:2006:Direct},
and Avron et al. use the Ball Pivoting algorithm 
from~\cite{Bernardini:1999:Ball}.
(Wang uses Poisson Surface Reconstruction described 
in~\cite{Kazhdan:2006:Poisson} but that algorithm
uses a variation of Marching Cubes.)
Neither Marching Cubes nor the Ball Pivoting algorithm
is particularly well-adapted to constructing meshes 
with good representations of sharp features.
By starting from the sharp features,
Schreiner et al. claim that their advancing front method
can do a good job of representing sharp features.




