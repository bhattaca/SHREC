
\firstsection{Introduction}

\maketitle

Given a regular grid sampling of a scalar field, $f: \Rthree \rightarrow \R$,
and a scalar value $\sigma$,
we are interested in constructing a good mesh representation
of the level set $f^{-1}(\sigma) = \{x : f(x) = \sigma \}$.
Such a representation is called an {\em isosurface} 
and the scalar value $\sigma$ is called an {\em isovalue}.

When $f$ is smooth,
i.e., when $f$ is continuous and its derivatives are continuous,
a number of algorithms do an excellent job of isosurface construction.
However, if the gradient field of $f$ is discontinuous,
then the problem of isosurface construction is more challenging.
If the level set $f^{-1}(\sigma)$ intersects a gradient discontinuity,
then the level set $f^{-1}(\sigma)$ will have a ``sharp corner''
(a 0-dimensional feature) or a ``sharp edge'' (a 1-dimensional feature)
at that discontinuity.
We are interested in constructing isosurfaces
which do a good job of representing those sharp features.

Isosurfaces are represented by piecewise linear or piecewise smooth meshes,
usually composed of triangles or quadrilaterals.
This underlying mesh should model the sharp features
of the isosurface $\Sigma$.
A 1-dimensionalf feature of $\Sigma$ with dihedral angle $\alpha$
should be represented by a single, 
connected sequence of mesh edges with dihedral angle near $\alpha$.
A 0-dimensional feature of $\Sigma$ with solid angle $\alpha$
should be represented by a single
isosurface vertex with solid angle near $\alpha$.
On the other hand, mesh edges and vertices
representing smooth, low curvature portions of $\Sigma$ should have
dihedral angles near 180 degrees and solid angles near $2 \pi$.

A number of algorithms have been proposed
for constructing isosurfaces 
with sharp features~\cite{ab-fpmmo-03,gk-eretm-04,hwco-cmsaf-05,
jlsw-dchd-02,kbsh-fssev-01,ms-ispmg-10,Varadhan:2003:fss,
sw-dmcpc-04,zhk-dctps-04}.
Unfortunately, these algorithms create ``notches'' along sharp edges,
degenerate, zero area triangles or quadrilaterals,
and ``folds'' in the mesh with ``flipped'' triangles.
(See Figures~??? in Section~??? for illustrations of these problems.)

In~\cite{bw-cisec-13},
we describe an algorithm, MergeSharp,
for constructing isosurfaces with sharp features
based on merging grid cubes around sharp features.
By placing a single isosurface vertex in these merged regions,
MergeSharp significantly reduces problems of ``notches'',
degenerate mesh polygons and ``folds'' in the mesh.
While MergeSharp reduced the mesh problems,
it did not eliminate them, 
with many meshes still having
one or two locations with degenerate mesh polygons or folds in the mesh.

In this paper, we present an algorithm, SHREC,
which almost completely eliminates the mesh problems listed above.
The algorithm is based on the MergeSharp paradigm,
where grid cubes around sharp features are merged
into a region containing a single isosurface vertex.
However, it differs from MergeSharp by better generation
of isosurface vertices on sharp features,
better selection of isosurface vertices on the sharp features,
and more controlled merging of grid cubes.
The algorithm performs measurably better
than MergeSharp or than other algorithms for which we had implementations.

There is extensive work on construction of surfaces with sharp features 
from point cloud data,
e.g.~\cite{avron2010L,cdr-drpsc-07,Daniels:2007:Robust,Dey2012,
fcs-rmlsf-2005,Oztireli2009,sym-fpmg-10,Wang:2013:Feature}
and many other articles.
In~\cite{Oztireli2009} and~\cite{Wang:2013:Feature},
the final construction of the surface mesh is accomplished 
by Marching Cubes~\cite{lc-mchr3-87}
or some other isosurface reconstruction algorithm.
The algorithms in this and similar papers
can be used by algorithms in those papers
to construct a surface mesh which better represents sharp surface features
than the Marching Cubes isosurface.

In~\cite{cdr-drpsc-07,Dey2012} and~\cite{sym-fpmg-10},
the final construction of the surface mesh is accomplished
by Voronoi based algorithms described 
in~\cite{cdr-drpsc-07} and \cite{Dey2012}.
These algorithms could be applied to isosurface reconstruction
from regular grid scalar data
by constructing point clouds from the grid data
and then applying the algorithms to the point clouds.
We compare isosurfaces with sharp features constructed
using the algorithm from~\cite{Dey2012}
with isosurfaces constructed by our algorithm.

Few of the isosurface or point cloud reconstruction papers
provide quantitative analysis of the resulting surfaces.
Some provide a distance measure (Hausdorff or root-mean squared)
between the constructed surface and an ideal surface,
but such measures fail to capture errors in surface normals
of the mesh polygons.
Instead of quantitative measures,
most papers provide a few images for visual inspection
of the quality of reconstructed features.
Needless to say, the lack of quantitative measures makes it
difficult evaluate the quality of the algorithms results,
to compare different algorithms,
or to comprehensively test an algorithm on a large number of data sets.

In contrast, the Hausdorff metric and variants are excellent
quantitative measures of the difference between reconstructed
and ideal smooth surfaces.
Software tools such as Metro~\cite{Cignoni:1998:metro} 
and MESH~\cite{Aspert:2002:MESH} are commonly used to measure
the quality of surface reconstructions
and provide quantitative evaluations of reported results.
However, these tools completely ignore surface orientation
and are not suitable for evaluating reconstructed features.

In this paper, we present the angular distance as a measure
of the difference between the surface orientation of two meshes.
Essentially, the angular distance measures the difference 
between the surface normal at a point on one mesh with the surface normals
on nearby points on the other mesh.
The angular distance is defined and discussed
in Section~\ref{section:angular_distance}.
We also provide a software tool which measures the angular distance.

In this paper, we are using the term ``feature'' to refer 
to a ``significant'' discontinuity in surface normals,
not a region with higher curvature.
Most of the papers on reconstruction from point clouds
don't distinguish between discontinuities in surface normals
and surface points with high curvature.
When isosurfaces are reconstructed from scalar data on a regular grid,
the sampling density is fixed and
it is impossible to determine if the original surface has
a discontinuity in surface normals or very high curvature.
Thus, we assume that features in our ideal surface are
points or curves where the surface normal is discontinuous.


\subsection*{Contributions}

The major contibution of our work is an isosurface generation algorithm
called SHREC which does measurable better than previous algorithms
in constructing isosurfaces with sharp features.
In particular, SHREC does better at generating and selecting vertices 
on sharp features and in avoiding constructing degenerate 
or small angle triangles and flipping triangles.
A second contribution is a definition of angular distance 
between two surfaces which we use to evaluate the quality 
of surface reconstruction algorithms.
