

\section{Related Work}
\label{section:related}

In~\cite{l-oslpm-00}, Lindtrom gives an algorithm for locating a point on a surface from a set of $n$ tangent planes.
The $n$ tangent give a set of $n$ equations in three unknowns,
described by $M x = b$ where $M$ is an $n \times 3$ matrix and $b$ is a column vector of length $n$.
Lindstrom approximates the intersection of the tangent planes using the singular valued decomposition (SVD) of $M$.
If all the eigenvalues of $M$ are large,
then approximate intersection of the tangent planes is the point
which is the least squares solution to $M x = b$
(or, equivalently, the solution to $M^T M x = M^T b$.)
If $M$ has two large eigenvalues, then the approximate intersection of the planes is a line.
Lindstrom's algorithm returns a point on that line closest to some input point.

The well-known Marching Cubes algorithm~\cite{lc-mchr3-87}
constructs isosurface patches within grid cubes.
The isosurface patches align along their common boundaries.
Because Marching Cubes positions vertices only on grid edges,
never inside grid cubes,
it does a poor job at representing sharp features on isosurfaces.

Kobbelt et al.~\cite{kbsh-fssev-01} modified Marching Cubes 
to position vertices inside grid cubes when those grid cubes
intersected sharp features.
In addition to scalar values at regular grid vertices,
the algorithm by Kobbelt et al. requires a directed distance field
representing the distance along each axis to the modeled surface.
The algorithm uses the directed distance field to determine
the location of vertices on sharp features.

Ju et al.~\cite{jlsw-dchd-02,sw-dcss-02} present a dual contouring algorithm
for constructing isosurfaces with sharp features.
Vertices of dual contouring isosurfaces lie inside grid cubes
while isosurface quadrilaterals are dual to grid edges.
Dual contouring algorithms were first described
by Gibson in~\cite{gh-ssqem-97,g-cesng-98}.
In addition to scalar values at regular grid vertices,
the algorithm by Ju et al. requires surface normals
at the intersection of the grid edges and the surface.
The surface normals are used to construct tangent planes
intersecting a grid cube.
The intersection of those tangent planes is approximated 
using least squares and singular valued decomposition,
and the resulting point is used as the isosurface vertex location.
Variations on~\cite{jlsw-dchd-02} are given 
in~\cite{zhk-dctps-04,Varadhan:2003:fss}.

Ashida and Bandler~\cite{ab-fpmmo-03}, Ho et al.~\cite{hwco-cmsaf-05}
and Gre{\ss} and Klein~\cite{gk-eretm-04} give algorithms
for constructing multiresolution isosurface with sharp features
using oct-trees or kd-trees.
The isosurface mesh is constructed
by first constructing polygonal curves representing the intersection
of the isosurface and oct-tree or kd-tree cells,
and then connecting an isosurface vertex with those polgyonal curves.
As in~\cite{jlsw-dchd-02},
isosurface locations are computed from input surface normals
using least squares and singular valued decomposition.

The Dual Marching Cubes algorithm by Schaefer and Warren~\cite{sw-dmcpc-04}
constructs a dual mesh which aligns with sharp features
and then extracts the isosurface from that mesh using Marching Cubes.
Because the isosurface is extracted using Marching Cubes,
the isosurface will have many "sliver" triangles with small angles.
Dual Marching Cubes reduces the number of "sliver" triangles
by positioning the vertices of the dual grid to lie on the isosurface,
whenever possible.

