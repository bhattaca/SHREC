
\section{Conclusion and Future Work}
\label{section:conclusion}

SHREC produces far fewer polygons with normal errors 
than any other software we tested,
but it still occasionally produces such errors.
Because of the regular structure of the regular grid,
we were hoping, but unable, to eliminate all such errors
and even to prove ``correctness'' of the reconstruction
under appropriate conditions.
We still believe that some algorithm along the lines of SHREC
should be able to construct isosurfaces with provable
guarantees on the reconstruction of sharp features
and surface normal directions.

One of the problems with the work on sharp feature reconstruction
is the lack of quantitative measures of the accuracy 
of the feature reconstruction.
We hope that the extensive quantitative comparisons in this paper
will set a precedent for such measurements in future work
on sharp feature reconstruction.

We presented the angle distance as a measurement of the difference
between the normals in two surfaces.
The angle distance does not obey the triangle inequality
and is not a metric.
It would be nice to have some measurement similar to angle distance
which measured the difference between surface normals
and is a metric.


\section{Acknowledgments}

We would like to thank Craig Leffel, Brent Obermiller 
and Honda of America Manufacturing
for providing the CMM and MotorCycle Engine datasets.
We would like also like to thank Christoph Heinzl
from the University of Applied Sciences Upper Austria
for providing the industrial CT volt dataset.

